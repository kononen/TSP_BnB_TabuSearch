\documentclass[12pt, a4paper]{article}

\usepackage[utf8]{inputenc}
\usepackage[T2A]{fontenc}
\usepackage[russian]{babel}
\usepackage{float}
\usepackage[oglav,spisok,boldsect,eqwhole,figwhole,hyperprint]{style}

\renewcommand{\labelenumi}{\arabic{enumi})}

\title{}
\author{А.\,А.~Кононенко}
\supervisor{И.\,К.~Марчевский}
\group{ФН2-42Б}
\date{2021}

\begin{document}
	
	\maketitle
	
	\tableofcontents
	
	\newpage
	
	\section-{История задачи} /*Введение*/
	
	В рамках прикладных проблем, которыми занимается «Комбинаторная оптимизация» (область теоретической математики), особой известностью пользуется «Задача коммивояжёра» (или TSP от англ. Travelling salesman problem), ее вариация «Задача о ходе коня» известна еще по работе Леонарда Эйлера 1759 г. 
	
	В общем виде «Задача коммивояжера» может быть сформулирована так:
	найти самый выгодный маршрут, проходящий через указанные города хотя бы по одному разу с последующим возвратом в исходный город.
	
	Критерии «выгодности» - могут определяться различно, например: минимизация длинны всего пути, времени в пути, стоимости в пути.
	
	Для формального определения задачи используется математическое определение «модели на графе». Граф определен множеством вершин, и множеством пар вершин – рёбер. Где вершины графа соответствуют городам, а рёбра между вершинами — пути сообщения между этими городами. Каждому ребру ставится в соответствие коэффициент выгодности.
	
	Для решения этой задачи могут использоваться алгоритмы разной сложности:
	Простые, но не сильно эффективные:
	\begin{enumerate}
		\item	полный перебор
		\item	случайный перебор
		\item	метод ближайшего соседа
		\item	метод включения ближайшего города
		\item	метод самого дешёвого включения
		\item	метод минимального остовного дерева
		\item	метод имитации отжига
 	\end{enumerate}	
		Эвристические, более эффективные:
	\begin{enumerate}
		\item	метод ветвей и границ,
		\item	метод генетических алгоритмов, 
		\item	алгоритм муравьиной колонии
	\end{enumerate}
	
	Рассмотрим подробно два метода:
	\begin{enumerate}
	\item	«Метод ветвей и границ», предложенный в 1960г. Алисой Лэнд и Элисон Дойг для решения комбинаторных задач, и применённый группой авторов к решению задачи Коммивояжёра в 1963г (Дж. Литл, К. Мурти, Д. Суини, К. Кэрол).
	
	\item	«Поиск с запретами», мета-алгоритм поиска, созданный Фредом У. Гловером в 1986 (формализованный им в 1989).
	\end{enumerate}
	
	\newpage
	
	\section-{Метод ветвей и границ}
	
	Метод ветвей и границ является эвристическим методом, который в отличии от полного перебора отсекает прохождение заведомо неоптимальных ветвей.
	
	Будем рассматривать в нашей задаче модельный полносвязный $n$-вершинный ориентированный граф, у которого между каждой ($i, j$) парой вершин существует 2 дуги с разной стоимостью проезда и движение одностороннее ($ C_{i j} \ne C_{j i}$).
	Решением задачи коммивояжёра является отыскание на нашем орграфе маршрута, проходящего однократно через все (n) вершин, при наименьшей его стоимости. 
	
	Проиллюстрируем решение задачи коммивояжера на простом графе с 4-мя вершинами:
	
	\begin{figure}[!h]
		\centering
		\includegraphics[scale=1]{graph1.pdf}
		\caption{График роста собственных значений $\lambda_n$ при $a=1$.}\label{speed}	
	\end{figure}
	
	Дерево потенциальных маршрутов для графа на 4-х маршрутах будет выглядеть следующим образом:
	
	
	\begin{figure}[!h]
		\centering
		\includegraphics[scale=1]{graph1.pdf}
		\caption{График роста собственных значений $\lambda_n$ при $a=1$.}\label{speed}	
	\end{figure}
	
	\newpage
	
	\begin{thebibliography}{9}
		\bibitem{Loyts} Лойцянский Л.\,Г. Механика жидкости и газа. М.: Дрофа, 2003. 846 с.
	\end{thebibliography}
	
\end{document}